% !TEX root = ../thesis-example.tex
%
\chapter{The socio-spatial stratification in cities}
\label{sec:concepts}


\begin{flushright}{\slshape    
The limits of my language\\
Are the limits of my world.} \\ \medskip
--- Ludwig Wittgenstein 
\end{flushright}

\section{Introduction}
\label{sec:introduction}

\section{Null model: the unsegregated city}
\label{sec:null_model_the_unsegregated_city}



There is often a negative connotation implied with the use of the word
`segregation', due to its use in relatively recent history. The use of the
`random city' as a null model however pushes forward the fact that segregation,
is, in fact 

We cannot judge the spatial repartition of people. There is no criterion of
`good' or `bad' in the way people arrange themselves, no moral values attached 
to any spatial pattern. It is the \emph{processes} that lead to such
patterns, the intentions behind people's decisions that make segregation
condemnable. It is the \emph{consequences} of segregation that may make
undesirable, something worth fighting against. Although it a priori seems that
the former aspect are a topic for qualitative research, being able to follow
households and study their pattern of relocation would also help in studying
the processes behind segregation. Schelling's cellular automata
model~\cite{Schelling:1971}, although intellectually stimulating, is very
limited in terms of predictions. A study of the dynamics of households
(spatially but also in terms of the evolution of their characteristics) should
deliver some precious insights about how individuals interact with their
surroundings.


\graffito{This remark sprung during a conversation with Viktoria Spaiser at
Uppsala University}

\section{Measuring the attraction and repulsion of categories}
\label{sec:measuring_the_attraction_and_repulsion_of_categories}

\section{The emergent social classes}
\label{sec:the_emergent_social_classes}

\section{Clustering and concentration}
\label{sec:clustering_and_concentration}

\section{Poor centers, rich suburbs?}
\label{sec:poor_centers_rich_suburbs_}

\section{A new measure of segregation}
\label{sec:a_new_measure_of_segregation}
