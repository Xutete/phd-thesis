% !TEX root = ../thesis-example.tex
%
\chapter{The socio-spatial stratification in cities}
\label{sec:concepts}


\begin{flushright}{\slshape    
The limits of my language\\
Are the limits of my world.} \\ \medskip
--- Ludwig Wittgenstein 
\end{flushright}

\section{Introduction}
\label{sec:introduction}

About models: \cite{Gauvin:2013, Brueckner:1999, Glaeser:2008}

\begin{quote}
    [Segregation] is a concept rich in theoretical suggestiveness and of
    unquestionable heuristic value. Clearly we would not wish to sacrifice the
    capital of theoretization and observation already invested in the concept.
    Yet this is what is involved in the solution offered by naive
    operationalism, in more or less arbitrary matching some convenient numerical
    procedure with the verbal concept of segregation... (Duncan and Duncan, 1955)
\end{quote}

\section{What segregation is not: the unsegregated city}
\label{sec:null_model_the_unsegregated_city}

There is often a negative connotation implied with the use of the word
`segregation', due to its use in relatively recent history. The use of the
`random city' as a null model however pushes forward the fact that segregation,
is, in fact, nothing but the deviation from a purely random spatial repartition
of individuals. What is unclear, is the reasons behind this deviation. Is it
because of an attraction to similar individuals, or a repulsion towards
individuals that are different from us? Or maybe is it just simple economic
mechanisms that are self-reinforced over time? If anything, the relatively
recent process of gentrification shows that the answer is not as simple as it
may have seemed at first: wealthy households moving to run-down nighbourhoods
tends to show that. Although
simplistic, Schelling's model of segregation shows that segregation does not
necessarily stem from evil intentions.

We cannot judge the spatial repartition of people. There is no criterion of
`good' or `bad' in the way people arrange themselves, no moral values attached 
to any spatial pattern. It is the \emph{processes} that lead to such
patterns, the intentions behind people's decisions that make segregation
condemnable. It is the \emph{consequences} of segregation that may make
undesirable, something worth fighting against. Although it a priori seems that
the former aspect are a topic for qualitative research, being able to follow
households and study their pattern of relocation would also help in studying
the processes behind segregation. Schelling's cellular automata
model~\cite{Schelling:1971}, although intellectually stimulating, is very
limited in terms of predictions. A study of the dynamics of households
(spatially but also in terms of the evolution of their characteristics) should
deliver some precious insights about how individuals interact with their
surroundings.


\graffito{This remark sprung during a conversation with Viktoria Spaiser at
Uppsala University}

\section{Measuring the attraction and repulsion of categories}
\label{sec:measuring_the_attraction_and_repulsion_of_categories}

\section{The emergent social classes}
\label{sec:the_emergent_social_classes}

\section{Clustering and concentration}
\label{sec:clustering_and_concentration}

\section{Poor centers, rich suburbs?}
\label{sec:poor_centers_rich_suburbs_}

\section{A new measure of segregation}
\label{sec:a_new_measure_of_segregation}
