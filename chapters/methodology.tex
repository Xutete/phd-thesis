% Introduction

\pdfbookmark[1]{Introduction}{Introduction}

\chapter{Methodology}
\label{chap:methodology}

\begin{flushright}{\slshape    
If it disagrees with experiment, it's wrong.\\ 
In that simple statement is the key to science.} \\ \medskip
--- Richard Feynman 
\end{flushright}

\bigskip

\section{The importance of being naive}
\label{sec:the_importance_of_being_naive}

\section{In defense of reductionism}
\label{sec:in_defense_of_reductionism}

In fact, we do not need to have a full theoretical description of the human mind
to say something about the actions of hundreds of thousands, a million of them.
In the same way we do not need string theory in order to explain the
functioning of living organisms.

Randomness is not to be understood as the opposite of rationality. Individuals
may as well make perfectly rational, predictable decisions, we would still have
to use a probabilistic approach. Particles are an extreme example of individuals
with a rational behaviour, yet we describe their collective behaviour with
statistical physics. Probabilities are not called for by the unpredictability of
one's behaviour, but rather by the particular type of information that is
important to us -- and that we can process.

What tells us that the world is not more complex than the picture drawn by
physical theories? That our best theories are only approximations that work a t
a certain scale, but are plain wrong at others? Everything around us, and the
history of physics itself. Reductionism does not imply renouncing to the world
in its entirity and its complexity. Reductionism is merely a recognition of our
limited capabilities, our possibility to grasp only the world tiny bit by tiny
bit, approximation after approximation. It is not absurb to reduce the amount of
information that is dealt with in theories, because this seems to be exactly how
we are cognitively programmed to function. If our brains were able to embrace
the world in all its details at once, one wouldn't need models, one would not
need theories, one wouldn't need science. Observation would be synonymous 
understanding.\\
History of Physics tends to prove to us that, in fact, all theories \emph{are}
effective theories -- in the sense that they are only true at a given scale. Too
aproximate when our measufing apparatus are able to probe nature more into
details. The analogy here is clear: we need more data, more specific data first
if we want to dig deeper into the reality of our societies, economies. We need
more than our own eyes, more than our ears. We need social, economical
telescopes.

\section{Quantitative stands for 'data'}
\label{sec:quantitative_stands_for_data_}

\section{Against data}
\label{sec:against_data}

Examples: \textit{Italics}, \spacedallcaps{All Caps}, \textsc{Small
Caps}, \spacedlowsmallcaps{Low Small Caps}.

\graffito{An example of side note}
In `Againt Method', the philosopher of science Paul Feyerabend argued against
the idea that Science proceeds through the application of a single, monolithic
method; what people usually call `The Scientific Method'. The reference is not
innocent, and I will argue here that, although empirical analysis constitutes
the alpha and the omega of our enquiry for knowledge, data are not enough.  

\section{Of models and theories}

\subsection{Why bother?}
\label{sub:why_bother_}

As scientific sceptics often like to remind us, all models, all theories are
wrong. But surely, there must be some interest in models to make them deserve
the months, sometimes years of work that scientist devote to them. Admittedly,
many take for granted the usefulness for model, or it seems so obvious that they
never question their quest.

The models two main functions are , broadly speaking, to understand, and to
predict.  The notion of simplification is close to the notion of `mental
economy' defended by the Philosopher of Science Ernst Mach.  According to him,
models' main function is to allow us to condense an infinity of possible
situations in very compact descriptions. Take Snell's law of refraction between
two media of optical indices $n_1$ and $n_2$

\begin{equation}
    n_1\,\sin \theta_1 = n_2\,\sin \theta_2
\end{equation}

In a single formula are contained an infinite number of possible experimental
configurations. Through this model, we are able

This picture is obviously incomplete. Some models obviously do not aim at being
correct from the beginning.

\subsection{Theory, not analogy}
\label{sub:theory_not_analogy}




There are two issues with the current state of affair 
Fancy words are too often used as a substitute for a real
understanding of the system. But, however intellectually appealing they are,
metaphors are not a theory. What do we understand from the comparison of cities
with biological systems? What new knowledge do we gain? Sure, the most important
ideas are those that you trigger, but you are responsible for the way your ideas
are interpreted.  Maybe cities are an easy pray for this sort of behaviour.
Since there is very little understandin, and the immediacy, complexity and
diversity of out experiences with cities creates this big gap in which anyone
can slip an idea or two, that will be interpreted in very different ways by
different people. The genius of metaphors is not to provide interesting ideas
that are ready to be applied to a specific fiedld. Rather, it is to trigger very
different ideas into different people. But this when we should realise the power
of the metaphor and leave it behind, following the trail of the new ideas thus
triggered. What we need to highlight are regularities, not similarities.

What is wrong, and somwehat uncomfortable in the present literature, is the
impression that models and reality live in two distinct, very loosely connected
worlds. Often, the gap is bridged through some intellectual trick, be it analogy
or metaphor.

A very nice quote is the beginning of the EPR paper

In the following manuscript, I will therefore pay a special attention to the
rigour in the language used. Qualify suggestions, by presenting them as such.
This kind of work may be less suggestive, the vocabulary used less expressive,
it may not make the reader feel as good about herself, but it is a necessary
step towards a science of cities. We need to clear the language of unfruitful
metaphors and fill the gap with mechanisms.
