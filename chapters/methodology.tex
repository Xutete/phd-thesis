% Introduction

\pdfbookmark[1]{Introduction}{Introduction}

\chapter{Methodology}
\label{chap:methodology}

\begin{flushright}{\slshape    
If it disagrees with experiment, it's wrong.\\ 
In that simple statement is the key to science.} \\ \medskip
--- Richard Feynman 
\end{flushright}

\section{The importance of being naive}
\label{sec:the_importance_of_being_naive}

\section{In defense of reductionism}
\label{sec:in_defense_of_reductionism}

In fact, we do not need to have a full theoretical description of the human mind
to say something about the actions of hundreds of thousands, a million of them.
In the same way we do not need string theory in order to explain the
functioning of living organisms.

\section{Quantitative stands for 'data'}
\label{sec:quantitative_stands_for_data_}

\section{Against data}
\label{sec:against_data}

Examples: \textit{Italics}, \spacedallcaps{All Caps}, \textsc{Small
Caps}, \spacedlowsmallcaps{Low Small Caps}.

\graffito{An example of side note}
In `Againt Method', the philosopher of science Paul Feyerabend argued against
the idea that Science proceeds through the application of a single, monolithic
method; what people usually call `The Scientific Method'. The reference is not
innocent, and I will argue here that, although empirical analysis constitutes
the alpha and the omega of our enquiry for knowledge, data are not enough.  
