\chapter{Studying cities}
\label{sec:studying_cities}

Walk a few steps in your favourite city, feel the streets bustling around you.
The sound of the cars, of people chatting, the architectural details of the
building standing in front of you, the mess---but not too much---on the
pavement. This familiarity we feel when stepping back in a city that was once
our home, a feeling of order. And that smell you had forgotten you knew. Maybe
the hardest thing, when studying cities, is the impression we have to know them
closely. The belief that our impression of what they are, the way we experience
them, gives a true picture of what they really are, the purpose they serve. We cannot properly know all the details of the life of a friend,
but only paint in our mind a picture of how they work from the tiniest details
we can gather. In the same way, we cannot infer what cities are solely from our
own experience as a user. We are a single piece of a puzzle that counts hundreds
of thousands, millions of them, all with a different opinion of what their
environment is like. No, to understand cities, how they work as a system, we
need to be told these thousands of stories, we need to analyse them and see
whether some are similar, how dissimilar they really are. We need to see whether
from the mess we can infer some general properties.\\

Cities appeared some $10,000$ years ago~\cite{Bairoch, Mumford}, concomitantly
with agriculture---although it is not clear which caused which, if there is such
a causal relation.

Cities are systems, a set of patterns and behaviours that make them identifiable
as such~\cite{Dennett:1991}. They are, in a sense, paradigmatic examples of
complex systems~\cite{Ladyman:2013}. Indeed, they comprise thousands, millions
of individuals that are moving and interacting constantly. Cities are indeed
more than the mere agglomeration of residences, factories and shops in the same
region; they exist and thrive through the resulting facilitated interaction
between individuals. 

A first step in the identification of order, is to identify the different
spatial and temporal scales at which 

