\chapter{Studying cities}
\label{sec:studying_cities}

% Cities very lively, we have the impression to know them but it's not enough.
% This is why we need data, information that is not biaised by our own way to
% experience things.
Walk a few steps in your favourite city, feel the streets bustling around you.
The sound of the cars, of people chatting, the architectural details of the
building standing in front of you, the mess---but not too much---on the
pavement. This familiarity we feel when stepping back in a city that was once
our home, a feeling of order. And that smell you had forgotten you knew. Maybe
the hardest thing, when studying cities, is the impression that we know them
closely. The belief that our impression of what they are, the way we experience
them, gives a true picture of what they really are, the purpose they serve. In a
sense, this is what makes the study of macroscopic, human-made systems so
difficult compared to the study of natural systems. There are only so many ways
we can get acquainted with electrons, and therefore just so many things about
them. Think about cities now, all the things you know about them. In a way, we
know too many things about them to be able to understand them. There is too much
information, that is not organised well enough to give us a sense of
understanding. If this was not enough, the large amount of disorganised
information we have about these systems is definitely biaised by our own
experience. We cannot properly know all the details of the life of a friend,
but only paint in our mind a picture of how they work from the tiniest details
we can gather. In the same way, we cannot infer what cities are solely from our
own experience as a user. We are a single piece of a puzzle that counts hundreds
of thousands, millions of them, all with a different opinion of what their
environment is like. No, to understand cities, how they work as a system, we
need to be told these thousands of stories, we need to analyse them and see
whether some are similar, how dissimilar they really are. We need data.\\

% Cities are a global issue, that is going to be here for a long time
Cities appeared some $10,000$ years ago~\cite{Bairoch, Mumford}, concomitantly
with agriculture---although it is not clear which caused which, if there is such
a causal relation. Although it is not completely clear why they exist, we can be
pretty sure they not just an accident in human history, and that they are going
to be around for a while. Indeed, born during the agriculture revolution,
cities really started to thrive after the industrial revolution~\cite{Bairoch}.
In England first, where the revolution was born, and London who was the first
cities in the modern world to reach $1,000,000$ inhabitants. The urban growth
then slowly spread through the end of the $19$th and the $20$th to the rest of
the Western world in the infancy of globalisation. Now, while most western
countries are almost completely urbanised (give numbers here), most of what has
been dubbed the 'urban \graffito{Source:\\ UN Population Division (2011)}
revolution' is happening in developing countries. The symbolic barrier has been
reached in $2005$, when it was estimated that more than $50\%$ \\

Why should we care about urban systems, then? Because
\begin{itemize}
    \item We can understand them
    \item We have data about them
    \item They are the place where most of humanity is going to live. By
        improving the way cities work, we can hopefully make dramatic changes to
        the way people live (although I am not quite sure yet of how to bring
        new knowledge to the ground).
\end{itemize}

% Cities are complex systems
Cities are systems, a set of patterns and behaviours that make them identifiable
as such~\cite{Dennett:1991}. They are, in a sense, paradigmatic examples of
complex systems~\cite{Ladyman:2013}. First, they comprise thousands, millions
of individuals that are moving and interacting constantly. Cities are indeed
more than the mere agglomeration of residences, factories and shops in the same
region; they exist and thrive through the resulting facilitated interaction
between individuals. Second, cities are incredibly resilient systems. Although
cities have disappeared in the past, think of the likes of Dresden or Hiroshima,
completely burnt to ashes during WWII, yet born again today, filled with even
more life than they were before the war. Finally, what is of interest to those
who are studying cities is that, as systems, cities exhibit very particular
shapes and behaviours. [Show picture of the US or Europe from space]\\

% Different temporal and spatial scales at which they occur
A first step in the identification of order, is to identify the different
spatial and temporal scales at which phenomena are occuring. The goal of any
theory of how cities work would be to understand the phenomena occuring at each
scale, and link the phenomena occuring at different scales. Establish a
hierarchy of mechanism, as is the case in natural sciences~\cite{Simon:1962}.

% Ackowledge the past
Of course, the description I have given of cities over the last few pages will
sound obvious to the reader owing to the work of predecessors.
