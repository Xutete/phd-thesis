\chapter{Studying cities}
\label{chap:studying_cities}

\begin{flushright}{\slshape    
Chaos was the law of nature;\\
Order was the dream of man.} \\ \medskip
--- Henry Adams~\cite{Adams:1990}
\end{flushright}

\bigskip

% 1. Add some trivial numbers (weight in economics...)
%        Economics
%        Environment
%        Surface Area
%        Total population
%        Urban revolution trivialities
%        Energy consumption
% 2. Cities, paradigmatic of complex systems
% 3. Cities, knowledge biaised by own experience
% 4. Put some order -> scales
% Study of processes occuring at different spatial and time scales.
% Understand how these scales relate to one another. (cite Pumain!)
%    CITIES
%       * Local = densities
%       * Relational = relational information (flows)
%       * Infrastructure
%    SYSTEMS OF CITIES
%        * Local = population (Zipf), socio-economic (Scaling)
%        * Relational = flows of individuals
%        * Infrastructure

% Cities very lively, we have the impression to know them but it's not enough.
% This is why we need data, information that is not biaised by our own way to
% experience things.
Walk a few steps in your favourite city, feel the streets bustling around you.
The sound of the cars, of people chatting, the architectural details of the
building standing in front of you, the mess---but not too much---on the
pavement. This familiarity we feel when stepping back in a city that was once
our home, a feeling of order. And that smell you had forgotten you knew. Maybe
the hardest thing, when studying cities, is the impression that we know them
closely. The belief that our impression of what they are, the way we experience
them, gives a true picture of what they really are, the purpose they serve. In a
sense, this is what makes the study of macroscopic, human-made systems so
difficult compared to the study of natural systems. There are only so many ways
we can get acquainted with electrons, and therefore just so many things about
them. Think about cities now, all the things you know about them. In a way, we
know too many things about them to be able to understand them. There is too much
information, that is not organised well enough to give us a sense of
understanding. If this was not enough, the large amount of disorganised
information we have about these systems is definitely biaised by our own
experience. We cannot properly know all the details of the life of a friend,
but only paint in our mind a picture of how they work from the tiniest details
we can gather. In the same way, we cannot infer what cities are solely from our
own experience as a user. We are a single piece of a puzzle that counts hundreds
of thousands, millions of them, all with a different opinion of what their
environment is like. No, to understand cities, how they work as a system, we
need to be told these thousands of stories, we need to analyse them and see
whether some are similar, how dissimilar they really are. We need data.\\


To reach a quantitative understanding of cities, we need to ignore, to forget
everything we \emph{think} we know about cities. We need to close our eyes and
stop walking. We need to forget abou the sounds and overwhelling smells. And
from the superficial chaos, let order emerge. 
A parallel can again be drawn with physics. Imagine an atom, at least the
picture that modern physics has of it. Is it realiable? According to our best
measuring apparatuses and protocols, it indeed is. Now ask the sightly different
question: is it accurate, does it reflect what an electron \emph{really} is? The
question is purely rhethorical, because we cannot observe an atom with our own
eyes. Yet, it is not difficult to imagine some underlying structure to it. We
could imagine, for instance, that electrons are an arrangement of imbricated
spheres, a bit like russian dolls. Providing this is not forbidden by any
current physical theories, and untestable empirically. We could. But would that
be useful? Probably not: we can invoke Occam's razor if we are fancy academics,
but it does sound reasonable to only keep the features in our theories that are
necessary to account for the observed behaviour. Nothing in Science
that is about `Realism', or the world `as it really is'~\cite{VanFraassen}. The
`Hidden world' is non-existing to science until it has been probed, measured.
Here we have tried to forget the individual and who it really is. We have tried
to focus instead on observables, measurable quantities.\\

% Cities are a global issue, that is going to be here for a long time
Cities appeared some $10,000$ years ago~\cite{Bairoch, Mumford}, concomitantly
with agriculture---although it is not clear which caused which, if there is such
a causal relation. Although it is not completely clear why they exist, we can be
pretty sure they not just an accident in human history, and that they are going
to be around for a while. Indeed, born during the agriculture revolution,
cities really started to thrive after the industrial revolution~\cite{Bairoch}.
In England first, where the revolution was born, and London who was the first
cities in the modern world to reach $1,000,000$ inhabitants. The urban growth
then slowly spread through the end of the $19$th and the $20$th to the rest of
the Western world in the infancy of globalisation. Now, while most western
countries are almost completely urbanised (give numbers here), most of what has
been dubbed the 'urban \graffito{Source:\\ UN Population Division (2011)}
revolution' is happening in developing countries. The symbolic barrier has been
reached in $2005$, when it was estimated that more than $50\%$ \\

Why should we care about urban systems, then? Because
\begin{itemize}
    \item We can understand them
    \item We have data about them
    \item They are the place where most of humanity is going to live. By
        improving the way cities work, we can hopefully make dramatic changes to
        the way people live (although I am not quite sure yet of how to bring
        new knowledge to the ground).
\end{itemize}

% Cities are complex systems
Cities are systems, a set of patterns and behaviours that make them identifiable
as such~\cite{Dennett:1991}. They are, in a sense, paradigmatic examples of
complex systems~\cite{Ladyman:2013}. First, they comprise thousands, millions
of individuals that are moving and interacting constantly. Cities are indeed
more than the mere agglomeration of residences, factories and shops in the same
region; they exist and thrive through the resulting facilitated interaction
between individuals. Second, cities are incredibly resilient systems. Although
cities have disappeared in the past, think of the likes of Dresden or Hiroshima,
completely burnt to ashes during WWII, yet born again today, filled with even
more life than they were before the war. Finally, what is of interest to those
who are studying cities is that, as systems, cities exhibit very particular
shapes and behaviours. [Show picture of the US or Europe from space]\\

Organised randomness. Historically, physics had to deal with 1. Simple systems
with a few variables, and deterministic equation to decribe their dynamics. 2.
Weakly, locally interacting systems with a very large number of particules.
\cite{Parisi:1999}.



% Different temporal and spatial scales at which they occur
A first step in the identification of order, is to identify the different
spatial and temporal scales at which phenomena are occuring. The goal of any
theory of how cities work would be to understand the phenomena occuring at each
scale, and link the phenomena occuring at different scales. Establish a
hierarchy of mechanism, as is the case in natural sciences~\cite{Simon:1962}.

%Dificulties with social sciences
How can we explain, then, the difference between social sciences and natural
ciences? Say, between economics and physics. Why, if the reductionist approach
is also to function with social systems, don't we have anything nearly as trong
a result as in the natural sciences? Clearly not because people are less
intellectually capable of handling difficult problems. Quite the
contrary, actually. One thing that jumps to your face when overlooking tebooks
in physics and textbooks in Geography, or in economics, is the difficulty of the
questions asked. The natural world is a complex realm as well, and there is no
reason to believe that the social world should be any more complicated (even
free minds obey to the law of large numbers when put together! And who tells you
electrons do not have a free mind too?). But the questions physics asks are
ridiculously simple, in comparison to what social scientists ask. In a way,
physicists have this natural advantage that Nature is a somewhat less immediate
reality to us, it is easier to distanciate oneself from physical phenomena than
it is from social phenomena. The problem with social sciences (if we can call
this a problem), is to try to solve complex questions because they are
\emph{important to us}. To try to develop an analogy, imagine physicists in the
18th century (before Kepler, before Newton...) were visited by an alien
civilisation, who would have described to them all the merits of the laser,
showed them one, and explained in what ways it could help us solve some basic
problems. Imagine they left, without leaving any explanation as to how build
one, or of the mecanisms involved behind the laser doing what it is doing.
Imagine politicians, pushed by the general public, decided to spend a lot of
money trying to build a laser. In what state do you think physics would be right
now? I am ready to bet no one would have never managed to build a laser, but we
we would have competing theories of what very general processes are at stake in
the processes of building a laser. Different schools of thoughts would have
emerged, urging to assemble this and this together. Until one day, maybe,
someone would have realised that we do not understand anything and have to go
back from basics, and forget about the laser for a time.
You only ever reach the goals you set to yourself.
Clearly, in some fields, economics being one of them, sociolog closely following
despite Bourdieu's criticism of the link between politics and sociology, this is
exactly what happening. Because they are asked to `make more money', `save the
conomy' or to justify such and such policy, these sciences have been preoccupied
with very high level, very complex questions that could be subdivided into tens,
hundreds, even more, simpler, more specific questions. They were asked to build
a rocket too soon. Successful approaches (with a few exceptions, see
thermodynamics) in physics were bottom-up approaches: by knowing the mecanisms
behind different phenomena, we were able to influence them, and combine them to
make objects that are more and more intricate. Top-down approaches rarely work,
because our brains are somehow not wired not handle very complex questions. But
we are cursed in the field of social sciences, because we are aware of the
top-down kind of questions, while you cannot aim at inventing a rocket before
knowing the basics of gravity. At least not seriously. There is something very
paradoxical in today's society in the very simple fact that while very complex
questions (regulate the economy, bringing full employment...) get massive
attention from funding agencies, while science-fiction-like projects such as
`inventing teletransportation' do not get a single euro from government
agencies. For me, they are of the same order. A wish to go towards an ideal,
allegedly better state of humanity, while not even knowing the mecanims on which
we would need to play in order to get there. While not even knowing
theoretically how to handle the problem. We only fund hard science projects that
we know are feasible, that are just a matter of technicalities, at least in
principle. Someone who would come up with the vague idea of writing a theory of
everything without any serious basis to think this is what he is doing would
have no chance to get any funding.\\

Jane Jacosb p 489: the order is complex systems exists once we have studied it.
Order emerges from chaos because of understanding. We cannot impose order in
cities in arbitrary way, we need to understand how they work first. We can
visualise cities as a very large systems with many retroaction loops. Planning
needs to understand what these loops are, and be careful of not destroying any
of them if it wants to be succesfull. Rather than imposing a structure on cities
and the way its inhabitants interact, it should rather observe what is
happening, understand why things work where they do work, why they do not work
where they do not work. Data analysis, modeling is a way of doing that.

Take, for instance, scaffolding. Scaffolding has a simple structure, made of
poles erected vertically, and horizontal boards attached to the poles that provide a
working surface for those using the scaffold. Scaffolding is simple to
understand, because we know how to build it, fitting tubes and boards together.
Now, imagine a leaf, with vascular bundles coming out of the stem and given the
leaft its particular shape. Leaves, obviously, have structure. They are not a
completely random assembly of the molecules that form the leaf. The structure of
leaves, however, is more difficult to comprehend than that of scaffolding.
Indeed, leaves are the result of many biological processes that lead to their
particular structure. And understanding why a leave has a particular shape
requires a lot of work on trying to uncover the relevant processes. 
Now, take cities. Cities are much more complicated than leaves.


Vascular systems are built once and for all and, if anything, only expand with
the body's size changing. On the other hand, cities adapt to the traffic on the
road which conversely is the result of the city's particular configuration.
Road evolve, buildings are built and destroyed, activities and people relocate,
etc. Cities are the delicate result of this interplay between transportation
networks and the activities, and a careful study of the former might lead to
invaluable insights into the structure and evolution of cities.

% Ackowledge the past
Of course, the description I have given of cities over the last few pages will
sound obvious to the reader owing to the work of predecessors.
