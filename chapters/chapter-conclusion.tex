% !TEX root = ../thesis-example.tex
%
\chapter{Conclusion}
\label{sec:conclusion}

\begin{flushright}{\slshape    
If people never did silly things\\
nothing intelligent would ever get done.} \\ \medskip
--- Ludwig Wittgenstein 
\end{flushright}


\section{What the past 3 years have brought}
\label{sec:what_the_past_3_years_have_brought}


What I find striking is the propention of the different communities to ignore
one another. Although there is a lot to say about physicists being ignorant of
most of the literature in social sciences, there is also a lot to say about
social sciences applying tools as black boxes, without any reflexion on the
meaning of the formula that is being applied. Despite an important literature on
the topic, we can still see exponents of power-law distribution estimated using
the Least-Squares method. A lot of the litterature on complex network is also
applied carelessly, a big victim being the algorithms of community detection (no
reflection on the meaning of communities, for instance). It is a real shame,
because we need people with quantitative skills to manipulate formulas and
create new measures, but we also need qualitative skills to manipulate concept
and interpret the significance of the results. While we cannot expect a
physicist or a computer scientitist to know all the literature in a field that
is not his own, we cannot expect social scientists to have strong quantitative
skills. The real solution resides in the collaboration.
\section{Limitations}
\label{sec:limitations}


The temptation is great, having a look back on $3$ years of work with a more
experienced eye, to understate the contributions of this thesis and their
potential applications. Not that I would do anything differently, but because I
now realise the work that is yet to be accomplished---and that is only for what
I know should be done, knowing that new problems and questions are ready to pop
out of nowehere at any time.

\section{Future work}
\label{sec:future_work}

I do believe there is a cruel lack of serious empirical work in the field. This
should be quickly solved, thanks to the information technologies that are now
available. But there is a bigger problem looming over our heads. The
uncomfortable fact that our fundamental object, the city, is an ill-defined
object. And that most empirical studies possibly rely on definitions of a city
that are not suited to the study they undertake. 
This lack of serious definition compromises the comparison between cities of
different countries (as I saw in the fingerprint paper), or different points in
time. I am, of course, not the first person to acknowledge this empirical
difficulty. In fact, it has been a long-time worry of geographers who have been
trying to produce harmonised database for a long time (cite
Bretagnolle, Pumain, any reference on harmonised database). Yet, we still lack
of an unambiguous, theoretically grounded definition of what a city is. And this
is problematic, since statistical institutes' results are based on what is
believed to be the best definition of the city at a time, which influences the
results, etc.
The problem is that it is unlikely there is only one way to define the object
city, but rather different definitions that depend on what we are studying.
People are usually confused by the existence of different definitions.


