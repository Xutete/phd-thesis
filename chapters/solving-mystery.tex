% !TEX root = ../thesis-example.tex
%
\chapter{The mystery of urban hierarchy}
\label{sec:concepts}

\begin{flushright}{\slshape    
The usual complaint about economic theory is that our models are
oversimplified---that offer excessively neat views of complex, messy reality.
[..] In one important case, the reverse is true: we have complex, messy models,
yet reality is startingly neat and simple.
} \\ \medskip
--- Paul Krugman~\cite{Krugman:1996}
\end{flushright}

\section{Introduction}
\label{sec:introduction}

\cite{Krugman:1996}
\cite{Gabaix:1999}
\cite{Batty:2013} for issues with Gibrat.
\cite{Cristelli:2012}
\cite{Batty:2006}

\section{The model}
\label{sec:the_model}


We write that 

\begin{equation}
    \partial_t P = \left(1-b-d\right)\,P + M_i^U + M_i^E
\end{equation} 

If the population is increasing wuch that

\begin{equation}
    P_{t+1} = \gamma_t\,P_t + \epsilon_t
\end{equation}

where $\epsilon_t$ is a small random increment with
$\mathrm{E}\left[\epsilon_t]\right] = \overline{\epsilon} > 0$. Whatever the
initial distirbution, the city-size distribution converges to a power-law with
and exponent $\zeta$ such that $\mathrm{E}\left[\gamma^\zeta\right] = 1$ (cite
Sornette). For small values of $\overline{\epsilon}$ we have

\begin{equation}
    \zeta = 1 + O(\epsilon)
\end{equation}

If the random variable is log-normally distributed, we have

\begin{equation}
    \zeta = 1 - \frac{2}{\sigma^2} \ln\left(1 -
    \frac{\overline{\epsilon}}{\overline{P}}\right)
\end{equation}

where $\overline{P}$ is the average size of a city. Therefore, the Zipf exponent
reads

\begin{equation}
    \boxed{    \alpha = \frac{1}{1 - \frac{2}{\sigma^2} \ln\left(1 -
\frac{\overline{\epsilon}}{\overline{P}}\right)}}
\end{equation}

From this equation one can clearly see that $\alpha > 1$ if and only if
$\overline{\epsilon} < 0$, that is when there is on average an outward flow
towards the outside of the system of cities (rural areas). On the other hand, we
have $\alpha < 1$ when $\overline{\epsilon} < 1$, that is when an important
rural exodus is happening, and $\alpha = 1$ if and only if $\overline{\epsilon}
= 0$, that is when the system of cities is at equilibrium with the rural
environment.
