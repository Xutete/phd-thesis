\chapter{What segregation is not}
\label{chap:segregation_introduction}

\begin{flushright}{\slshape    
The limits of my language\\
Mean the limits of my world.} \\ \medskip
--- Ludwig Wittgenstein~\cite{Wittgenstein:1998}
\end{flushright}


\bigskip


% Defining average and variance
\newcommand{\E}{\mathrm{E}}
\newcommand{\Var}{\mathrm{Var}}

\section{Studying segregation}
\label{sec:studying_segregation}


We cannot judge the spatial repartition of people. There is no criterion of
`good' or `bad' for the way people arrange themselves, no moral values attached
to any spatial pattern. It is the \emph{processes} that lead to such patterns,
the intentions behind people's decisions that make segregation condemnable. It
is the \emph{consequences} of segregation that may make it undesirable, something
worth fighting against.\\

% consequences
% need for a local information
As a matter of fact, social residential segregation has terrible consequences.
As shown in~\cite{Massey:1993}, residential segregation is the cause of major
economic disadvantages that affect the least affluent segments of the
population, through the isolation from social networks, or the presence of
deficient public service in the poorest areas. Worse, it has been shown that
increased levels of segregation in urban areas is associated with a higher
mortality burden~\cite{Lobmayer:2002}. For all these reasons, there is a
somewhat urgent need to measure the extent of segregation, especially its local
component, and understand the underlying mechanisms.\\

In the literature, authors systematically design a single index of segregation for
territories that can be very large, up to thousands of square
kilometers~\cite{Apparicio:2000}. In order to mitigate segregation, a more
local, spatial information is however needed: local authorities need to locate
where the poorest and richest concentrate if they want to design efficient
policies to curb, or compensate for, the existing segregation. In other words,
we need to provide a clear {\it spatial} information on the pattern of
segregation. We need to identify the areas where levels of segregation are high.

Besides, if we want to design policy or incentives to reduce socio-spatial
stratification and its consequences, we need to understand the processes at
play. We need to understand why segregation patterns exist, and why they
persist. Without mechanistic insights, attempts at regulating segregation may
have unforeseen, possibly damaging consequences.  The processes behind
segregation are however unclear. Schelling's cellular automata
model~\cite{Schelling:1971}, although intellectually stimulating, is very
limited in terms of predictions. More sophisticated models appeared
recently~\cite{Brueckner:1999,Glaeser:2008,Gauvin:2013}, yet the link with the
empirical reality is too thin, and processes are yet to be validated.  

In fact, we believe that the lack of an appropriate model is likely due to the lack
of identification of a clear structure, or clear behaviours in the data. In
order to identify the processes at play, we urgently 
need to properly describe the spatial patterns of segregation; the
dynamics of households (how they move, how their characteristics evolve over
time) and neighbourhoods (how their population changes).\\

In the following, we will therefore focus on the empirical characterisation of
the patterns of segregation. But first, we need to define what we mean when we
talk about residential segregation.

\section{Think first, measure later}
\label{sec:introduction}

% the curse of familiarity 
As stated many times, and at different periods in the sociology
literature~\cite{Duncan:1955m,James:1982,Massey:1988,Reardon:2002}, the study of
segregation is cursed by its intuitive appeal. Pretty much everyone has heard of
segregation, and has an opinion about it. This familiarity with the concept
favours what Duncan and Duncan~\cite{Duncan:1955m} called `naive operationalism':
the tendency to force a sociological interpretation on measures that are at odds
with the conceptual understanding of segregation. In their own words

\begin{quote}
    [Segregation] is a concept rich in theoretical suggestiveness and of
    unquestionable heuristic value. Clearly we would not wish to sacrifice the
    capital of theoretization and observation already invested in the concept.
    Yet this is what is involved in the solution offered by naive
    operationalism, in more or less arbitrary matching some convenient numerical
    procedure with the verbal concept of segregation... (Duncan and Duncan,
    1955~\cite{Duncan:1955m})
\end{quote}

For all its intuitive appeal, segregation is however an intricate, compound
notion whose complexity only reveals itself through careful study. However
tempting it is to start writing measures of segregation that seem `reasonable',
it is necessary to stop and think about the meaning of the notion first. We need to
\emph{think} segregation to be able to provide \emph{useful} measures of
segregation.


\section{The dimensions of segregation}
\label{sec:the_dimensions_of_segregation}

Segregation has been extensively studied in the Sociology and Geography
literature. The most important conceptual heritage of this literature is the
distinction between residential segregation's different dimensions.
Massey~\cite{Massey:1988} first proposed a list of $5$ dimensions (and related
existing measures), which was recently reduced to $4$ by
Reardon~\cite{Reardon:2004}. 

\begin{description}
    \item[Evenness] (and clustering in the continuous
        limit, as shown by Reardon~\cite{Reardon:2004}) is the extent to which
        populations are evenly spread in the metropolitan area.
        Measures of evenness are affected by the fact that
        individuals are not spread uniformly across space in urban areas,
        disregarding of their respective category;

    \item[Exposure] is the extent to which different
        populations share the same residential area. This presupposes defining
        what is meant by `residential area';

    \item[Concentration] is the extent to which populations concentrate in their
        residential area;

    \item[Centralisation] is the extent to which populations concentrate in the
        center of the city. As we have seen in
        Chapter~\ref{chap:monocentric_introduction}, the notion of center
        is meaningless in large, polycentric urban areas;
\end{description}

We will discuss in details the shortcomings of the measures currently proposed
for each of these dimensions in Chapter~\ref{chap:patterns_segregation}.


\section{The unsegregated city}
\label{sec:null_model_the_unsegregated_city}

The fundamental issue with the picture given by these 4 dimensions lies in the
lack of a general theoretical framework in which all existing measures can be
interpreted.  Instead, we have a patchwork of seemingly unrelated measures that
are labelled with either of the aforementioned dimensions. Already in $1986$,
Michael White~\cite{White:1986} regretted the fact that segregation was never
defined in the literature, and always considered as a given. Each index implied a different definition of segregation, which lead to endless
debates about the virtues of such or such measure (dubbed the `index war').
Unknowingly, authors were trying to squeeze the social reality into existing
measures. When, in fact, one should start by defining the social reality, before
attempting to capture it with appropriate measures. As of today, no such
definition of segregation exists. We shall begin our study of segregation
patterns by an attempt at defining segregation. All the measure we propose then
naturally follow.\\

Segregation manifests itself in different ways, which makes it very difficult to
define. It is however easy to define what is \emph{not} segregation: a spatial
distribution of different categories that is undistinguishable from a uniform
random situation~\cite{Jahn:1947}. Therefore, we propose to define segregation
as the following

\begin{quote}
Segregation is any pattern in the spatial distribution of populations that
significantly deviates from a situation where individuals would have chosen
their residence at random (densities and overall category proportions being
equal).  
\end{quote} 

It is then easy to understand the different dimensions
of~\cite{Massey:1988,Reardon:2004}: each of the dimensions correspond to a
different ways in which a multi-dimensional pattern can deviate from its
randomized counterpart. Our definition is perfectly agnostic with regards to
the features of the population density pattern. It is also not concerned with
the overall inequality levels.\\

In the context of residential segregation in urban areas, a natural null model
is therefore the \emph{unsegregated city}. In the unsegregated city, all
households are distributed at random within the urban space with the further
constraints that

\begin{itemize}
    \item The total number $N_\alpha$ of people belonging to a category
	    $\alpha$ is fixed and equal to that found in the data;
    \item The total number $n(t)$ of households living in the areal unit $t$ is
	    fixed and equal to that found in the data.
\end{itemize}

which also fixes the total number of individuals $N$ in the city. The problem of
finding the numbers $\left( n_\alpha(1), \dots, n_\alpha(T) \right)$ of
individuals belonging to a certain category $\alpha$ in the $T$ areal units of
an unsegregated city is reminiscent of the traditionnal occupancy problem in
combinatorics~\cite{Feller:1950}. Their distribution is given by the multinomial
distribution $f \left( n_\alpha(1), \dots, n_\alpha(T) \right)$, and the number
of people of category $\alpha$ in the areal unit $t$ by a binomial distribution.
Therefore, in an unsegregated city, we have

\begin{align}
    \begin{split}
	\E \left[ n_\alpha(t) \right] &= N_\alpha\,\frac{n(t)}{N} \\
	\Var \left[ n_\alpha(t) \right] &= N_\alpha\,\frac{n(t)}{N} \left( 1 - \frac{n(t)}{N}  \right) 
    \end{split}
\end{align}

where $N$ is the total number of households in the city. In metropolitan areas
$N_\alpha$ is larged compared to $1$, and the distribution of the $n_\alpha(t)$
can be approximated by a Gaussian with the same mean and variance.


Most studies exploring the question of spatial segregation define measures
before comparing their value for different cities. Knowing that two quantities
are different is however not enough: we also have to know whether this
difference is {\em significant}. In order to assess the significance of a
result, we have to compare it to what is obtained for a reasonable null model.
As we will see in Chapter~\ref{chap:patterns_segregation}, the 
unsegregated city model allows us to assess whether a given pattern is
the result of a segregation process or not.

Any spatial distribution patterns could theoretically obtained via a random
repartition of households. They are however not equally likely. We propose to
measure the total segregation by the likelihood of obtaining a given pattern,
assuming a random distribution.

\bigskip

In this chapter, we have discussed some of the improvements that could be
brought to the existing measures in the literature. In particular, we have
emphasized the need for a \emph{local} knowledge of the patterns of segregation.
We have also laid the theoretical foundation upon which we are going to design
new measures.
In Chapter~\ref{chap:patterns_segregation}, we start from the above-defined null
model to propose a way to quantify the presence of various categories in parts
of the city. This allows us to identify and delineate neighbourhoods, measure the interactions
between the categories, and extract a class structure from the spatial pattern
alone.
