% !TEX root = ../main.tex
%
\chapter{The (end of the) monocentric city}
\label{sec:related}


\section{Introduction}
\label{sec:introduction}

[Picture 3D? of employment density profiles of different cities]


Clark in $1951$ plots the population density of various cities as a function of
population size.~\cite{Clark:1951}
Why monocentric? Because it is a very simple assumptions and allows simple
analysis. Check whether Clark does mention von Th\"unen or not, etc.\\


Show: Density profile of a strongly polycentric city and anisotropic
(Minneapolis), and the percolation threshold at which it is not monocentric any
more.

The problem with the traditional models of cities in two pictures: the
(polycentric) reality and the AMM model.


The Alonso-Muth-Mills model might well be the reason for the long-lasting influence of
the monocentric model (a nice exposition of the model can be found
in~\cite{Fujita:1989}). In 1964, Alonso introduced the bid-rent curve as a
function of the distance to the city center~\cite{Alonso:1964}. The simplifying hypothesis of
\emph{monocentricity} naturally followed, the assumption that all firms in a
city are concentrated in a single, fixed-size part of the city, the central business
district. We should not underestimate how this model influenced many people's
perception of what a city is. In the US, the name of Central Business District
is casually used as a way to designate the principle activity center in a city.
Many, if not most, measures of the spatial variation of quantities inside cities
actually use the notion of `distance to the city center'. This only makes sense,
however, under the assumption of monocentricity. Most of the
literature has not departed from the monocentric assumption---sometimes without
being aware of it! In the defense of these authors however, there is a clear
lack of appropriate tools to study spatial profiles.


\cite{Mills:1972} is a monograph discussing the causes of decentralisation and
suburbanisation. [could not read]

\cite{Kemper:1974} explores data trying to fit a negative exponential function
to industry and employment density. It does not work so well.

\cite{Odland:1978} explores the possibility of polycentric cities on a
theoretical basis. [could not read]

\cite{Griffith:1981} tool to evaluate the polycentricity of a pattern.[could not
read]

Treatment of the polycentric city in the urban economics literature starts
with~\cite{Fujita:1982}.

\cite{Dokmeci:1994} shows that Istanbul's employment is spread across several
centers. Although there is still a lack of strong quantitative evidence, the
idea is gaining ground.


\section{How to measure polycentrity}
\label{sec:how_to_measure_polycentrity}

\cite{McDonald:1987} proposes the first empirical method to identify employment
subcenters.

\cite{Giuliano:1991} uses an arbitrary threshold to determine the centers.

\cite{Anas:1998} Critices the method of Giuliano, has a cool picture of
Los-Angeles spreading and deals with urban structure.

\cite{McMillen:2001} proposes a non-parametric method to find the subcenters.

\cite{McMillen:2003} proposes a pretty good literature review in introduction on
the identification of employment subcenters, and proposes congestion as a reason
for the polycentric transition.

\cite{Tsai:2005} is a classic.

\cite{Griffith:2007} is yet another reference on spatial regression.

\cite{Pereira:2013} proposes an Urban Centrality Index that varies continuously
between a monocentric configuration and an extreme decentralized situation.

\cite{Louf:2013_polycentric} Proposes to use a property of the rank plot of the
employment density, exponential decrease.

\cite{Louail:2014} proposes a generalisation of the previous method based on the
Lorenz curve.

\cite{LeNechet:2015} a more recent paper.

I propose a generalisation, based on the same hypothesis but that is more robust
than the computation of the tangent at the origin.

Justify the term `polycentric transition'. When the population increases, it is
clear that the number of centers increases as well. Therefore, it seems that, as
they grow and expand, urban systems develop a more and more polycentric form.
