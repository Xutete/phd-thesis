% Introduction

\pdfbookmark[1]{Outline}{Outline}

\chapter{About this thesis}
\label{chap:methodology}

%\begin{flushright}{\slshape    
%In the long run men only hit what they aim at.\\
%Therefore, though they should fail immediately\\
%they had better aim at something high.} \\ \medskip
%--- Henry David Thoreau
%\end{flushright}

\begin{flushright}{\slshape    
Anybody can plan weird, that's easy.} \\ \medskip
--- Charles Mingus
\end{flushright}


\bigskip 


The following thesis might surprise the reader used to the monographs usually
produced by PhD students in Social Sciences, articulated around a single,
general question. The outline of this thesis reflects more the
line of thoughts and of research that has been undertaken than the answer to a
single question that would have been asked a priori and answered during the
last three years. For that reason, the four Parts of this thesis are mostly
independent. There is not single thread holding them together. But rather multiple
wires; common themes and similar ideas. 

\section{Outline}

Part~\ref{part:polycentricity} tackles the problem of measuring and
understanding urban form, an issue that has been running through the $3$ years
of my PhD. In this Part, we first (Chapter~\ref{chap:monocentric_introduction})
present a brief historical overview of the monocentric and polycentric
representations of the city, before enumerating the methods that are used in the
literature to count the number of activity centers. We end with the observation
that the number of activity centers increases in a regular way with population
size. The following chapter (Chapter~\ref{chap:monocentric_model}) is devoted to
an out-of-equilibrium model that we built in order to explain the previous
empirical regularity. The model is able to predict the sublinear increase of the
number of centers that we observe on American and Spanish data. In the last
chapter (Chapter~\ref{chap:monocentric_discussion}), we question the assumptions
of the model and the current empirical methods to quantify urban form.\\

Part~\ref{part:scaling} is concerned with scaling relationships. We first propose
(Chapter~\ref{chap:scaling_introduction}) a non-exhaustive overview of the dawn
and surge of allometric scalings, from Stewart's $1949$ to the recent wealth of
studies. Then, using the model developped in the preceding part, we show in
Chapter~\ref{chap:scaling_model} how the structure of mobility patterns allow us
to understand the qualitative and quantitative values of the exponents related
to urban form and mobility. We conclude this part with a discussion on the
interpretation of these scaling laws, and their important shortcomings
(Chapter~\ref{chap:scaling_implications}).\\

Part~\ref{part:segregation} departs from the preceding chapters and turns to the
study of residential segregation. Driven by the desire to extend the model
presented in Chapter~\ref{chap:monocentric_model}, we soon realised there was a
lack of robust empirical description of patterns of segregation that could be
reproduced by a model. In Chapter~\ref{chap:segregation_introduction} we tackle
the problem of defining what segregation is; we propose a brief review of the
existing literature, and subsequently define a null model -- the segregated
city. In the next chapter (Chapter~\ref{chap:patterns_segregation}), we build on
this null model to propose a set of measures to quantify patterns of residential
segregation.\\

Part~\ref{part:networks} concerns the original topic of this thesis: spatial
networks. Because my interests have shifted towards the study of
socio-economical phenomena over the years, we only briefly present the most
important results in the present thesis. The three chapters are, for the most
part, reprints of articles that have been previously published in peer-reviewed
journals. We first (Chapter~\ref{chap:typology}) present an empirical study of
$131$ street patterns across the world where we propose a method to classify the
patterns based on the geometrical shape of the blocks. In the following chapter
(Chapter~\ref{chap:cost-benefit}), we present a cost-benefit analysis framework
to understand the properties and growth of spatial networks.  We introduce an
iterative model that can explain the emergence of a hierarchical structure
(`hubs and spokes') in growing spatial networks. Starting from the cost-benefit
framework of this model, we show that the length, number of stations and
ridership of subways and rail networks can be estimated knowing the area,
population and wealth of the underlying region.\\

Finally, Part~\ref{part:conclusion} ties everything together, highlights the lessons
learned and concludes this thesis with some potentially interesting research avenues for the
years to come.



\section{Miscellaneous notes}

\subsection{Style}
\label{sub:style}

I will be using the pronoun 'we' for most of the manuscript, to reflect the fact
that the work presented here was, for the most part, done in the context of
collaboration with others. For the sake of clarity, the technical details of
calculations have been omitted in this manuscript. Most of these calculations are
relatively simple anyway, and the interested reader can find them in the
publications mentioned on page~\ref{p:publications} of this thesis.

\subsection{Tools}
\label{sub:tools}

Unless otherwise specified, all figures in this manuscript have been prepared
using Python $2.7$~\footnote{Available at \url{http://www.python.org}} and the
Matplotlib library~\cite{Hunter:2007}. Inkscape~\footnote{Available at
\url{https://inkscape.org/en/}} was used to prepare most diagrams. This document
was typeset using Vim and \LaTeX. The template used is the typographical look-and-feel
\texttt{classicthesis} developed by Andr\'e
Miede.\footnote{Available at 
\url{http://code.google.com/p/classicthesis/}.}

\begin{center}
\end{center}
