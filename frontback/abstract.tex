% Abstract

\pdfbookmark[1]{Abstract}{Abstract} % Bookmark name visible in a PDF viewer

\begingroup
\let\clearpage\relax
\let\cleardoublepage\relax
\let\cleardoublepage\relax

\chapter*{Abstract} % Abstract name


The amount of data that is being gathered about cities is increasing in size and
specificity. However, despite this wealth of information we still have little
understanding of what really drives the processes behind urbanisation. In this
thesis we apply some ideas from statistical physics to the study of cities, in
an attempt to expand the amount of understanding we have of these systems.\\ 



% Monocentric to polycentric transition
We present a stochastic, out-of-equilibrium model of city growth that describes
the structure of the mobility pattern of individuals. The model explains the
appearance of secondary subcenters as an effect of traffic congestion. We are
also able to predict the sublinear increase of the number of centers with
population size, a prediction that is verified on American and Spanish data. 

% Scaling
Within the framework of this model, we are further able to give a prediction for the
scaling exponent of the total distance commuted daily, the total length of the
road network, the total delay due to congestion, the quantity of
CO\textsubscript{2} emitted, and the surface area with the population size of
cities. Predictions that agree with data gathered for U.S. cities.

% Segregation
In the third and last part, we focus on the quantitative description of the
patterns of residential segregation. We propose a unifying theoretical framework
in which segregation can be empirically characterised. We propose a measure of
interaction between the different categories. Building on the information about
the attraction and repulsion between categories, we are able to define classes
in a quantitative, unambiguous way. The framework also allows us to identify the
neighbourhoods where the different classes concentrate, and characterise their
properties and spatial arrangement. Finally, we revisit the traditional
dichotomy between poor city center and rich suburbs and provide a measure that
is adapted to anisotropic, polycentric cities.\\


Throughout this thesis, we try to convey the idea that the complexity of cities is --
almost paradoxically -- better comprehended through simple approaches. 
Looking for structure in data, trying to isolate the most important processes,
building simple models and only keeping those which agree with data, constitute
a universal method that is relevant to the study of any system.


\endgroup			

\vfill
